%% template.tex
%% from
%% bare_conf.tex
%% V1.4b
%% 2015/08/26
%% by Michael Shell
%% See:
%% http://www.michaelshell.org/
%% for current contact information.
%%
%% This is a skeleton file demonstrating the use of IEEEtran.cls
%% (requires IEEEtran.cls version 1.8b or later) with an IEEE
%% conference paper.
%%
%% Support sites:
%% http://www.michaelshell.org/tex/ieeetran/
%% http://www.ctan.org/pkg/ieeetran
%% and
%% http://www.ieee.org/

%%*************************************************************************
%% Legal Notice:
%% This code is offered as-is without any warranty either expressed or
%% implied; without even the implied warranty of MERCHANTABILITY or
%% FITNESS FOR A PARTICULAR PURPOSE!
%% User assumes all risk.
%% In no event shall the IEEE or any contributor to this code be liable for
%% any damages or losses, including, but not limited to, incidental,
%% consequential, or any other damages, resulting from the use or misuse
%% of any information contained here.
%%
%% All comments are the opinions of their respective authors and are not
%% necessarily endorsed by the IEEE.
%%
%% This work is distributed under the LaTeX Project Public License (LPPL)
%% ( http://www.latex-project.org/ ) version 1.3, and may be freely used,
%% distributed and modified. A copy of the LPPL, version 1.3, is included
%% in the base LaTeX documentation of all distributions of LaTeX released
%% 2003/12/01 or later.
%% Retain all contribution notices and credits.
%% ** Modified files should be clearly indicated as such, including  **
%% ** renaming them and changing author support contact information. **
%%*************************************************************************


% *** Authors should verify (and, if needed, correct) their LaTeX system  ***
% *** with the testflow diagnostic prior to trusting their LaTeX platform ***
% *** with production work. The IEEE's font choices and paper sizes can   ***
% *** trigger bugs that do not appear when using other class files.       ***                          ***
% The testflow support page is at:
% http://www.michaelshell.org/tex/testflow/

\documentclass[conference,final,]{IEEEtran}
% Some Computer Society conferences also require the compsoc mode option,
% but others use the standard conference format.
%
% If IEEEtran.cls has not been installed into the LaTeX system files,
% manually specify the path to it like:
% \documentclass[conference]{../sty/IEEEtran}





% Some very useful LaTeX packages include:
% (uncomment the ones you want to load)


% *** MISC UTILITY PACKAGES ***
%
%\usepackage{ifpdf}
% Heiko Oberdiek's ifpdf.sty is very useful if you need conditional
% compilation based on whether the output is pdf or dvi.
% usage:
% \ifpdf
%   % pdf code
% \else
%   % dvi code
% \fi
% The latest version of ifpdf.sty can be obtained from:
% http://www.ctan.org/pkg/ifpdf
% Also, note that IEEEtran.cls V1.7 and later provides a builtin
% \ifCLASSINFOpdf conditional that works the same way.
% When switching from latex to pdflatex and vice-versa, the compiler may
% have to be run twice to clear warning/error messages.






% *** CITATION PACKAGES ***
%
%\usepackage{cite}
% cite.sty was written by Donald Arseneau
% V1.6 and later of IEEEtran pre-defines the format of the cite.sty package
% \cite{} output to follow that of the IEEE. Loading the cite package will
% result in citation numbers being automatically sorted and properly
% "compressed/ranged". e.g., [1], [9], [2], [7], [5], [6] without using
% cite.sty will become [1], [2], [5]--[7], [9] using cite.sty. cite.sty's
% \cite will automatically add leading space, if needed. Use cite.sty's
% noadjust option (cite.sty V3.8 and later) if you want to turn this off
% such as if a citation ever needs to be enclosed in parenthesis.
% cite.sty is already installed on most LaTeX systems. Be sure and use
% version 5.0 (2009-03-20) and later if using hyperref.sty.
% The latest version can be obtained at:
% http://www.ctan.org/pkg/cite
% The documentation is contained in the cite.sty file itself.






% *** GRAPHICS RELATED PACKAGES ***
%
\ifCLASSINFOpdf
  % \usepackage[pdftex]{graphicx}
  % declare the path(s) where your graphic files are
  % \graphicspath{{../pdf/}{../jpeg/}}
  % and their extensions so you won't have to specify these with
  % every instance of \includegraphics
  % \DeclareGraphicsExtensions{.pdf,.jpeg,.png}
\else
  % or other class option (dvipsone, dvipdf, if not using dvips). graphicx
  % will default to the driver specified in the system graphics.cfg if no
  % driver is specified.
  % \usepackage[dvips]{graphicx}
  % declare the path(s) where your graphic files are
  % \graphicspath{{../eps/}}
  % and their extensions so you won't have to specify these with
  % every instance of \includegraphics
  % \DeclareGraphicsExtensions{.eps}
\fi
% graphicx was written by David Carlisle and Sebastian Rahtz. It is
% required if you want graphics, photos, etc. graphicx.sty is already
% installed on most LaTeX systems. The latest version and documentation
% can be obtained at:
% http://www.ctan.org/pkg/graphicx
% Another good source of documentation is "Using Imported Graphics in
% LaTeX2e" by Keith Reckdahl which can be found at:
% http://www.ctan.org/pkg/epslatex
%
% latex, and pdflatex in dvi mode, support graphics in encapsulated
% postscript (.eps) format. pdflatex in pdf mode supports graphics
% in .pdf, .jpeg, .png and .mps (metapost) formats. Users should ensure
% that all non-photo figures use a vector format (.eps, .pdf, .mps) and
% not a bitmapped formats (.jpeg, .png). The IEEE frowns on bitmapped formats
% which can result in "jaggedy"/blurry rendering of lines and letters as
% well as large increases in file sizes.
%
% You can find documentation about the pdfTeX application at:
% http://www.tug.org/applications/pdftex





% *** MATH PACKAGES ***
%
%\usepackage{amsmath}
% A popular package from the American Mathematical Society that provides
% many useful and powerful commands for dealing with mathematics.
%
% Note that the amsmath package sets \interdisplaylinepenalty to 10000
% thus preventing page breaks from occurring within multiline equations. Use:
%\interdisplaylinepenalty=2500
% after loading amsmath to restore such page breaks as IEEEtran.cls normally
% does. amsmath.sty is already installed on most LaTeX systems. The latest
% version and documentation can be obtained at:
% http://www.ctan.org/pkg/amsmath





% *** SPECIALIZED LIST PACKAGES ***
%
%\usepackage{algorithmic}
% algorithmic.sty was written by Peter Williams and Rogerio Brito.
% This package provides an algorithmic environment fo describing algorithms.
% You can use the algorithmic environment in-text or within a figure
% environment to provide for a floating algorithm. Do NOT use the algorithm
% floating environment provided by algorithm.sty (by the same authors) or
% algorithm2e.sty (by Christophe Fiorio) as the IEEE does not use dedicated
% algorithm float types and packages that provide these will not provide
% correct IEEE style captions. The latest version and documentation of
% algorithmic.sty can be obtained at:
% http://www.ctan.org/pkg/algorithms
% Also of interest may be the (relatively newer and more customizable)
% algorithmicx.sty package by Szasz Janos:
% http://www.ctan.org/pkg/algorithmicx




% *** ALIGNMENT PACKAGES ***
%
%\usepackage{array}
% Frank Mittelbach's and David Carlisle's array.sty patches and improves
% the standard LaTeX2e array and tabular environments to provide better
% appearance and additional user controls. As the default LaTeX2e table
% generation code is lacking to the point of almost being broken with
% respect to the quality of the end results, all users are strongly
% advised to use an enhanced (at the very least that provided by array.sty)
% set of table tools. array.sty is already installed on most systems. The
% latest version and documentation can be obtained at:
% http://www.ctan.org/pkg/array


% IEEEtran contains the IEEEeqnarray family of commands that can be used to
% generate multiline equations as well as matrices, tables, etc., of high
% quality.




% *** SUBFIGURE PACKAGES ***
%\ifCLASSOPTIONcompsoc
%  \usepackage[caption=false,font=normalsize,labelfont=sf,textfont=sf]{subfig}
%\else
%  \usepackage[caption=false,font=footnotesize]{subfig}
%\fi
% subfig.sty, written by Steven Douglas Cochran, is the modern replacement
% for subfigure.sty, the latter of which is no longer maintained and is
% incompatible with some LaTeX packages including fixltx2e. However,
% subfig.sty requires and automatically loads Axel Sommerfeldt's caption.sty
% which will override IEEEtran.cls' handling of captions and this will result
% in non-IEEE style figure/table captions. To prevent this problem, be sure
% and invoke subfig.sty's "caption=false" package option (available since
% subfig.sty version 1.3, 2005/06/28) as this is will preserve IEEEtran.cls
% handling of captions.
% Note that the Computer Society format requires a larger sans serif font
% than the serif footnote size font used in traditional IEEE formatting
% and thus the need to invoke different subfig.sty package options depending
% on whether compsoc mode has been enabled.
%
% The latest version and documentation of subfig.sty can be obtained at:
% http://www.ctan.org/pkg/subfig




% *** FLOAT PACKAGES ***
%

%\usepackage{fixltx2e}
% fixltx2e, the successor to the earlier fix2col.sty, was written by
% Frank Mittelbach and David Carlisle. This package corrects a few problems
% in the LaTeX2e kernel, the most notable of which is that in current
% LaTeX2e releases, the ordering of single and double column floats is not
% guaranteed to be preserved. Thus, an unpatched LaTeX2e can allow a
% single column figure to be placed prior to an earlier double column
% figure.
% Be aware that LaTeX2e kernels dated 2015 and later have fixltx2e.sty's
% corrections already built into the system in which case a warning will
% be issued if an attempt is made to load fixltx2e.sty as it is no longer
% needed.
% The latest version and documentation can be found at:
% http://www.ctan.org/pkg/fixltx2e


%\usepackage{stfloats}
% stfloats.sty was written by Sigitas Tolusis. This package gives LaTeX2e
% the ability to do double column floats at the bottom of the page as well
% as the top. (e.g., "\begin{figure*}[!b]" is not normally possible in
% LaTeX2e). It also provides a command:
%\fnbelowfloat
% to enable the placement of footnotes below bottom floats (the standard
% LaTeX2e kernel puts them above bottom floats). This is an invasive package
% which rewrites many portions of the LaTeX2e float routines. It may not work
% with other packages that modify the LaTeX2e float routines. The latest
% version and documentation can be obtained at:
% http://www.ctan.org/pkg/stfloats
% Do not use the stfloats baselinefloat ability as the IEEE does not allow
% \baselineskip to stretch. Authors submitting work to the IEEE should note
% that the IEEE rarely uses double column equations and that authors should try
% to avoid such use. Do not be tempted to use the cuted.sty or midfloat.sty
% packages (also by Sigitas Tolusis) as the IEEE does not format its papers in
% such ways.
% Do not attempt to use stfloats with fixltx2e as they are incompatible.
% Instead, use Morten Hogholm'a dblfloatfix which combines the features
% of both fixltx2e and stfloats:
%
% \usepackage{dblfloatfix}
% The latest version can be found at:
% http://www.ctan.org/pkg/dblfloatfix




% *** PDF, URL AND HYPERLINK PACKAGES ***
%
%\usepackage{url}
% url.sty was written by Donald Arseneau. It provides better support for
% handling and breaking URLs. url.sty is already installed on most LaTeX
% systems. The latest version and documentation can be obtained at:
% http://www.ctan.org/pkg/url
% Basically, \url{my_url_here}.




% *** Do not adjust lengths that control margins, column widths, etc. ***
% *** Do not use packages that alter fonts (such as pslatex).         ***
% There should be no need to do such things with IEEEtran.cls V1.6 and later.
% (Unless specifically asked to do so by the journal or conference you plan
% to submit to, of course. )



%% BEGIN MY ADDITIONS %%



\usepackage[unicode=true]{hyperref}

\hypersetup{
            pdftitle={Improving P2P Lending Returns: An Annotated Bibliography},
            pdfkeywords={P2P Lending},
            pdfborder={0 0 0},
            breaklinks=true}
\urlstyle{same}  % don't use monospace font for urls

% Pandoc toggle for numbering sections (defaults to be off)
\setcounter{secnumdepth}{0}
% Pandoc header
\interlinepenalty=10000
\widowpenalty=1000

%% END MY ADDITIONS %%


\hyphenation{op-tical net-works semi-conduc-tor}

\begin{document}
%
% paper title
% Titles are generally capitalized except for words such as a, an, and, as,
% at, but, by, for, in, nor, of, on, or, the, to and up, which are usually
% not capitalized unless they are the first or last word of the title.
% Linebreaks \\ can be used within to get better formatting as desired.
% Do not put math or special symbols in the title.
\title{Improving P2P Lending Returns: An Annotated Bibliography}

% author names and affiliations
% use a multiple column layout for up to three different
% affiliations

\author{
\IEEEauthorblockN{Ryan Kuhn}
\IEEEauthorblockA{Independent\\
\\
Basking Ridge, NJ 07920\\
kuhnrl30@gmail.com
}
}

% conference papers do not typically use \thanks and this command
% is locked out in conference mode. If really needed, such as for
% the acknowledgment of grants, issue a \IEEEoverridecommandlockouts
% after \documentclass

% for over three affiliations, or if they all won't fit within the width
% of the page, use this alternative format:
%
%\author{\IEEEauthorblockN{Michael Shell\IEEEauthorrefmark{1},
%Homer Simpson\IEEEauthorrefmark{2},
%James Kirk\IEEEauthorrefmark{3},
%Montgomery Scott\IEEEauthorrefmark{3} and
%Eldon Tyrell\IEEEauthorrefmark{4}}
%\IEEEauthorblockA{\IEEEauthorrefmark{1}School of Electrical and Computer Engineering\\
%Georgia Institute of Technology,
%Atlanta, Georgia 30332--0250\\ Email: see http://www.michaelshell.org/contact.html}
%\IEEEauthorblockA{\IEEEauthorrefmark{2}Twentieth Century Fox, Springfield, USA\\
%Email: homer@thesimpsons.com}
%\IEEEauthorblockA{\IEEEauthorrefmark{3}Starfleet Academy, San Francisco, California 96678-2391\\
%Telephone: (800) 555--1212, Fax: (888) 555--1212}
%\IEEEauthorblockA{\IEEEauthorrefmark{4}Tyrell Inc., 123 Replicant Street, Los Angeles, California 90210--4321}}




% use for special paper notices
%\IEEEspecialpapernotice{(Invited Paper)}




% make the title area
\maketitle

% As a general rule, do not put math, special symbols or citations
% in the abstract
\begin{abstract}
Current models suggest that Peer-to-Peer lending profitability can be
improved with the application of high quality data analysis. The
existing body of research generally focused on minimizing the
probability of default rather than optimizing for profitability. To
encourage the consolidation of knowledge, This bibliography is a source
of research and articles from around the web that could be used to
understand the current state of analysis.
\end{abstract}

% no keywords

% use for special paper notices



% make the title area
\maketitle

% no keywords

% For peer review papers, you can put extra information on the cover
% page as needed:
% \ifCLASSOPTIONpeerreview
% \begin{center} \bfseries EDICS Category: 3-BBND \end{center}
% \fi
%
% For peerreview papers, this IEEEtran command inserts a page break and
% creates the second title. It will be ignored for other modes.
\IEEEpeerreviewmaketitle


\section{Introduction}\label{introduction}

Peer-to-peer lending connects borrowers with investors through an online
marketplace. Loans offered through Lending Club are made possible by
investors, who invest in exchange for solid returns.

This bibliography classifies articles into one of three categories based
on their source and intented audience: academic research, industry
analysis, and models `in the wild'. Industry analysis generally explain
a narrowly focused insight through review of specific variable
relationships. Models use by individual investors may methodically
review loan attributes and propose further testing approaches.

\subsection{Academic Research}\label{academic-research}

\textbf{Ala'raj, Maher, and Maysam F. Abbod}, \textbf{2016}, \textbf{A
new hybrid ensemble credit scoring model based on classifiers consensus
system approach., }\textbf{\emph{Expert Systems with
Applications}}\textbf{ 64}\textbf{, 36--55}

\textbf{Cimpoeru, Smaranda Stoenescu}, \textbf{2011}, \textbf{Neural
networks networks and their application in credit risk assessment.
evidence from the romanian market., }\textbf{\emph{Technological \&
Economic Development of Economy}}\textbf{ 17}\textbf{, 519--534}

\textbf{Emekter, Riza et al.}, \textbf{2015}, \textbf{Evaluating credit
risk and loan performance in online peer-to-peer (p2p) lending.,
}\textbf{\emph{Applied Economics}}\textbf{ 47}\textbf{, 54--70}

\textbf{Keough, David, Nicolaus Enko, and Brian Shake},
\textbf{Developing a data science approach to detecting income fraud for
the peer to peer loan industry., }\textbf{\emph{SSRN}}\textbf{
(https://ssrn.com/abstract=2836134)}

An artile exploring the concept that fraudulently reported income
increases default risk since the borrowor was granted more credit than
their ability to repay warranted. The authors hypothesize that borrowers
with verified income will default less than non-verified borrowers but
their limited testing lead to the opposite conclusion. The superficial
analysis did not identify or control for confounding variables leaving
the conclusion unreliable.

\textbf{Ge, Ruyi et al.}, \textbf{2017}, \textbf{Predicting and
deterring default with social media information in peer-to-peer
lending., }\textbf{\emph{Journal of Management Information
Systems}}\textbf{ 34}\textbf{, 401--424}

\textbf{Guo, Yanhong et al.}, \textbf{2016}, \textbf{Instance-based
credit risk assessment for investment decisions in p2p lending.,
}\textbf{\emph{European Journal of Operational Research}}\textbf{
249}\textbf{, 417--426}

\textbf{Nandi, Jayanta Kishore, and Navin Kumar Choudhary},
\textbf{2011}, \textbf{Credit risk management of loan portfolios by
indian banks: Some empirical evidence., }\textbf{\emph{IUP Journal of
Bank Management}}\textbf{ 10}\textbf{, 32--42}

\textbf{Reddy, Sriharsha, and Krishna Gopalaraman}, \textbf{2016},
\textbf{Peer to peer lending, default prediction evidence from lending
club., }\textbf{\emph{Journal of Internet Banking \& Commerce}}\textbf{
21}\textbf{, 1--19}

\textbf{Serrano-Cinca, Carlos, and Begoña Gutièrrez-Nieto},
\textbf{2016}, \textbf{The use ofprofit scoring as an alternative to
credit scoring systems in peer-to-peer (p2p) lending,}\textbf{. Article}
\textbf{\emph{Decision Support Systems}}

This article proposes a decision support system for choosing loans
optimized for profitability. This is in contrase to previous literature
which focused on modeling default probabilities without considering the
increased interest rates to compensate the investor's risk. The authors
proposed using internal rate of return, or IRR, to measure of a loan's
profitability. IRR was chosen because it is able to handle uneven
payments at non-uniform intervals. They created several models predict a
loan's IRR as the dependent variable.

\subsection{Industry Analysis}\label{industry-analysis}

Orchard Mon Ja Lending Times Lend Academy Lending Robot NSR/ Peer Cube
Peer Lending Server

\textbf{Anon.}, \textbf{2015}, \textbf{Lending club - predicting loan
outcomes,}\textbf{. Blog} \textbf{\emph{Peer Lending Server}}\textbf{
(\href{Analyzing\%20Lending\%20Club\%20Payment\%20History}{Analyzing Lending Club Payment History})}

\textbf{Renton, Peter}, \textbf{2011}, \textbf{How lending club and
prosper set interest rates,}\textbf{. Blog} \textbf{\emph{Lend
Academy}}\textbf{
(\url{http://www.lendacademy.com/how-lending-club-and-prosper-set-interest-rates/})}

\textbf{Wu, James}, \textbf{2015}, \textbf{Changes in lending club
underwriting,}\textbf{. Blog} \textbf{\emph{MonJa}}\textbf{
(\url{http://www.monjaco.com/blog/changes-in-lending-club-underwriting-looking-beneath-the-headlines/})}

\subsection{\texorpdfstring{Modeling ``In the
Wild''}{Modeling In the Wild}}\label{modeling-in-the-wild}

\textbf{Cashorali, Tanya}, \textbf{2011}, \textbf{Mining lending club's
goldmine of loan data part i of ii visualizations by state,}\textbf{.
Blog} \textbf{\emph{R-bloggers}}\textbf{
(\url{https://www.r-bloggers.com/mining-lending-clubs-goldmine-of-loan-data-part-i-of-ii-visualizations-by-state/})}

This geographical analysis of loans prior to Q3 2011 showed a high
proportion of loans in California and Florida defaulted there whereas
Texas, Pennsylvania, and New Jersey had comparably low default rates.

\textbf{Darre, Jean-Francois}, \textbf{2015}, \textbf{Analysis of
lending club's data,}\textbf{. Blog} \textbf{\emph{Data Science
Central}}\textbf{
(\url{http://www.datasciencecentral.com/profiles/blogs/analysis-of-lending-club-s-data})}

An analysis exploring the relationship between many of the loan
attributes and the historical default experience. The author puts forth
the analysis as exploratory rather than seeking to answer a specific
questions. Hence, their are only observations and not specific
conclusions. The observations include: Default rates decrease with
credit history age; higher credit utilization leads to greater risk of
default; loan grades are progressively less correlated with the FICO
score, presumably, as LC's rating algorithm improved.

\textbf{Davenport, Kevin}, \textbf{2015}, \textbf{Lending club data
analysis revisited with python,}\textbf{. Blog} \textbf{\emph{Machine
Learning and Statistics Blog}}\textbf{
(\url{http://kldavenport.com/lending-club-data-analysis-revisted-with-python/})}

An analysis to understand which attributes are driving Lending Club's
interest rates. It includes a brief summary of the components of the
FICO score, such as length of credit history and payment history, and
how that may be a starting point for identifying confounding variables.
The discussion suggests finding local optima of credit risk to interest
rate given that LC uses flat interest rates for loan sub grades. The
article was written in 2013 so does not reflect the subsequent changes
in LC's scoring model.

\textbf{Davenport, Kevin}, \textbf{2013}, \textbf{Gradient boosting:
Analysis of lendingclub's data,}\textbf{. Blog}\textbf{
(\url{http://kldavenport.com/gradient-boosting-analysis-of-lendingclubs-data/})}

An expansion on the author's earlier analyis but with discussion around
challenges in improving the model. Challenges includes developing
meaningful groupings of employement title by overcoming the current
diversity in the dataset.

\textbf{Davis, Jason}, \textbf{2012}, \textbf{Lending club loan
analysis: Making money with logistic regression,}\textbf{. Blog}
\textbf{\emph{Data Startups}}\textbf{
(\url{http://drjasondavis.com/blog/2012/04/08/lending-club-loan-analysis-making-money-with-logistic-regression})}

This analysis is proposes a simple valuation method by weighting the
interest rate by the probability of defaulting to arrive at the expected
interest rate. The author used this valuation method along with logistic
regression model built to predict the default risk to show that returns
can be significantly improved for portfolios with less than 400 loans.

\textbf{Anon.}, \textbf{Give me some credit,}\textbf{. Blog}
\textbf{\emph{Kaggle}}\textbf{
(\url{https://www.kaggle.com/c/GiveMeSomeCredit})}

\textbf{Mistry, Hitesh}, \textbf{2016}, \textbf{Application of survival
analysis to p2p lending club loans data,}\textbf{. Blog}
\textbf{\emph{Systems Forecasting}}\textbf{
(\url{http://systemsforecasting.com/2016/07/application-of-survival-analysis-to-p2p-lending-club-loans-data/})}

An insightful analysis using 3 unique methods including: modeling
profitability with survival analysis, measuring significance with
concordance instead of p-values, and measuring profit as the ratio of
cash in/out flows. Survival analysis can measure to probability that the
cash inflows, or payment received, will reach payback a given threshold.
Concordance analysis is more robust to big data and prevents
statistically significant variables that don't improve the model.

\textbf{Moy, Philip}, \textbf{2015}, \textbf{Lending club default
analysis, }\textbf{
(\url{http://www.creditreportservice.info/article/95227030/lending-club-default-analysis/})}

\textbf{ORourke, Ted}, \textbf{2016}, \textbf{Lending club - predicting
loan outcomes,}\textbf{. Blog} \textbf{\emph{Rpubs}}\textbf{
(\url{https://rpubs.com/torourke97/190551})}

A blog post which quickly moves through a cursory exploratory analysis
and on to applying machine learning algorithms. O'Rourke applies two
algorithms to Lending Club's historical loan data from 2007-2011:
Decision Tree and Logistic Regression. The author has not developed a
hypothesis to test against nor contributed new insights to the body of
knowledge. The article is a common form of analysis meant to demonstrate
the author's ability to apply a model to a dataset which is a necessary
step in the learning process.

\textbf{Patierno, David}, \textbf{2011}, \textbf{Using genetic
algorithms to maximize lending club performance,}\textbf{. Blog}\textbf{
(\url{http://blog.dmpatierno.com/post/3161338411/lending-club-genetic-algorithm})}

Using a genetic algorithm, the author proposes a rules based approach
for selecting a mix of loans which maximizes the cash inflows.

\textbf{Polena, Michal}, \textbf{2016}, \textbf{Investing at lending
club with watson analytics,}\textbf{. Blog} \textbf{\emph{IBM}}\textbf{
(\url{https://www.ibm.com/blogs/business-analytics/lending-club-and-watson-analytics-1/})}

\textbf{Summers, Cameron}, \textbf{2016}, \textbf{Lending club deep
learning, }\textbf{ (\url{https://scaubrey.github.io/})}

This article takes the first steps in building a model to optimize
profitability but falls short by not including lost principle in returns
calculation. The `loss' of a defaulted loan is the opportunity cost of
lost interest and understates the downside risk for a given laon.

\textbf{Toth, Michael}, \textbf{2015}, \textbf{Analyzing historical
default rates of lending club notes,}\textbf{. Blog}
\textbf{\emph{R-bloggers}}\textbf{
(\url{http://michaeltoth.me/analyzing-historical-default-rates-of-lending-club-notes.html})}

An exploratory analysis of loan attributes and their connection to the
loan's outcome. Toth provides a step-by-step write up of how he analyzed
data from 2012-2013. The article does not include a summary nor a
discussion of the results. An interesting method was to group the loan
purposes into three categories: 1) debt, 2) major purposes such as home
improvement, and 3) luxury purchase such as vacations and weddings.

\textbf{Tsai, Kevin, Sivagami Ramiah, and Sudhanshu Singh},
\textbf{2014}, \textbf{Peer lending risk predictor, }\textbf{
(\href{http://cs229.stanford.edu/proj2014/Kevin\%20Tsai,Sivagami\%20Ramiah,Sudhanshu\%20Singh,Peer\%20Lending\%20Risk\%20Predictor.pdf}{http://cs229.stanford.edu/proj2014/Kevin Tsai,Sivagami Ramiah,Sudhanshu Singh,Peer Lending Risk Predictor.pdf})}

\end{document}


